\chapter{期中學習歷程}
\section{課程重點(一到十周))}

% 10周的calendar
\begin{calendar}{9/09/24}{10}
\setlength{\calboxdepth}{.3in}
\setlength{\calwidth}{\linewidth}
% Description of the Week.
\calday[Monday]{\classday} % Monday
\skipday \skipday \skipday \skipday % 跳過星期2345
\skipday\skipday % weekend (no class)
% 第1~10周的課程大綱
\caltext{9/9/24}{1. 課程簡介 2. 人工智慧簡介 3. 課程結構與評分方式 4. 課程導向與學習目標 5. 人工智慧的應用}
\caltext{9/16/24}{1. 臺南大學場域驗證影片 2. 艾倫圖靈的故事 3. AI在語言教育中的應用 4. 跨語言交流與文化保存 5. 知識圖產生}
\caltext{9/23/24}{1. AI-FML人機共學 2. 什麼是人工智慧 3. 知識本體論(Ontology) 4. 人類智慧與人工智慧的比較 5. 人工智慧模擬人類思考與行動的能力}
\caltext{9/30/24}{1. 機器學習 2. 深度學習 3. 生成式AI生成策略 4. 生成式 AI 的應用與特性 5. ChatGPT 的運作與語言模型}
\caltext{10/7/24}{1. AI競賽說明 2. 休閒旅遊推薦模型 3. 語意項函式 4. CI推論模型 5. 學習工具串接}
\caltext{10/14/24}{1. 使用生成式AI分析文章 2. 推動 AI 素養與跨領域知識整合 3. 生成式 AI 的虛擬知識與教育應用 4. 使用AI協助撰寫新聞稿 5. 精簡文本內容}
\caltext{10/21/24}{1. AI及人類新聞稿相似度對比 2. ELMO如何分類單字 3. Embedding 4. BERT 5. GPT應用}
\caltext{10/28/24}{1. 臺南大學 AI 機器人新聞知識圖 2. 臺英慧虛擬機器人對話 3. 圖片生成 4. 虛擬機器人的互動性 5. 體驗心得}
\caltext{11/4/24}{1. 期中作業 2. 上傳競賽資料集 3. 上傳推論模型 4. 不同模型參數比較 5. 知識圖比較}
\caltext{11/11/24}{1. 得獎同學上台報告 2. 陽明交通大學演講投影片製作成文章 3. 課程學習應用 4. 報告改進內容 5. 報告技術細節}

\end{calendar}

\section{量子AI競賽實作歷程}
\subsection{參賽背景}

2024 年,我參加了國立臺南大學與 IEEE 台北分會共同舉辦的「全國 AI 專題創意競賽」。這場競賽吸引了 40 支來自全國各校的參賽隊伍,最終 24 支隊伍進入決賽。我們的作品著重於生成式 AI 進行資料分析和擴增以及量子計算的結合,透過實現與生活相關的應用,並以優異表現榮獲冠軍。
\subsection{學習過程}

    技術應用與模型設計
    我們的解決方案以梯形函數作為語意項,結合 PSO(粒子群優化)演算法進行核心運算,旨在提升系統的精準度與效能。透過反覆模擬與測試,我們成功構建了一個穩定且準確率高的模型,充分展現了生成式 AI 與量子計算在實際問題中的潛力。

    挑戰與解決
    雖然準備過程中一切順利,但決賽當天的報告卻發生了一個意外:我們誤放了一個語意項函式的圖形,導致現場報告受到影響。這突如其來的狀況讓我們一度陷入困惑,但我們仍舊保持鎮定,以口頭闡述的方式補充說明,努力說服評審我們設計的完整性與創意性。

    團隊合作與改進
    團隊內部合作過程充滿挑戰。我們在比賽籌備期間以及課餘時間進行了多次討論,最終透過有效溝通與協作找到最佳解決方案。這不僅提高了團隊效率,也讓我們更加理解跨領域合作的重要性。

\subsection{比賽成果}

儘管決賽過程出現失誤,我們的模型的高準確性以及流暢的報告內容也讓我們成功贏得冠軍。這樣的結果讓我們既感到驚喜,也深感對未知充滿敬畏。這次經歷教會我們,成功不僅來自完美的準備,更源於臨場的應變能力與團隊的支持。
回顧整個參賽歷程,我深刻體會到技術實踐與問題解決的重要性,尤其是梯形函數與 PSO 演算法的創新應用,讓我們的設計在競爭中脫穎而出。另一方面,意外的發生也提醒我,學術與實踐並非總是一帆風順,臨場應變與抗壓能力同樣至關重要。
透過這次競賽,我對生成式 AI 與量子計算的未來應用充滿期待。未來,我希望能結合更多複雜語意模型與智慧優化演算法,開發更具價值的應用。
