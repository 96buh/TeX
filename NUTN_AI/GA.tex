% 演化式計算
\chapter{演化式計算和PSO}
\section{Genetic Algorithm}
基因演算法(Genetic Algorithm)是一種啟發式演算法,用於求解複雜的優化問題或尋找接近最佳解的近似解。他會模仿自然界生物的進化過程,包括遺傳、突變、交配(重組)和自然選擇等機制。這些特性使得 GA 能夠在大規模搜索空間中高效地進行全局搜尋。
\subsection{步驟}
\begin{enumerate}
    \item 初始化 \\
        先隨機產生一組初始解,稱為種族族群(Population)。每個解稱為個體(Individual),每個個體都有一組編碼稱為基因(Gene)。
    \item 適應度函數(Fitness Function) \\
        定義一個函數來評估每個個體的好壞,稱為適應度(Fitness)。適應度值反映了該解在問題中接近最佳解的程度。
    \item 選擇(Selection) \\
        擇出一些適應度較高的個體作為父代,使用方法包括輪盤賭選擇(Roulette Wheel Selection)或錦標賽選擇(Tournament Selection)。
    \item 交配(Crossover) \\
        將選中的父代進行基因重組,生成新的子代個體。常見交叉方式包括單點交叉(Single-point Crossover)和多點交叉(Multi-point Crossover)。
    \item 突變(Mutation) \\
        根據突變率(Mutation rate)隨機改變某些基因的值,以保持種群的多樣性並避免陷入局部最優。突變的概率通常較低。
    \item 重複 \\
        重複進行選擇、交配和突變,直到達到停止條件(如達到最大迭代次數或適應度達到某個閾值)。

% 流程圖 -------------------
\begin{figure}[ht]
\centering
\begin{tikzpicture}[node distance=3cm, auto, >=stealth, 
  every node/.style={draw, align=center, font=\footnotesize, minimum height=1cm, text centered, fill=gray!10}]

% 定義節點樣式
\tikzstyle{process} = [rectangle, rounded corners, minimum width=3cm, fill=gray!10]
\tikzstyle{decision} = [diamond, aspect=2, minimum width=3cm, fill=gray!20]
\tikzstyle{arrow} = [thick,->, >=stealth]

% 節點
\node (start) [process] {開始};
\node (init) [process, right of=start, xshift=0.5cm] {初始化種群};
\node (fitness1) [process, right of=init, xshift=0.5cm] {適應度評估};
\node (selection) [process, right of=fitness1, xshift=0.5cm] {選擇};
\node (crossover) [process, right of=selection, xshift=0.5cm] {交配 (Crossover)};
\node (mutation) [process, below of=crossover, yshift=1cm] {變異 (Mutation)};
\node (fitness2) [process, below of=mutation, yshift=1cm] {適應度評估 (新一代)};
\node (nextgen) [process, left of=fitness2, xshift=-1.2cm] {選擇下一代};
\node (decision) [decision, left of=nextgen, xshift=-1.5cm] {是否滿足終止條件?};
\node (output) [process, left of=decision, xshift=-2.25cm] {輸出最優解};

% 連接箭頭
\draw [arrow] (start) -- (init);
\draw [arrow] (init) -- (fitness1);
\draw [arrow] (fitness1) -- (selection);
\draw [arrow] (selection) -- (crossover);
\draw [arrow] (crossover) -- (mutation);
\draw [arrow] (mutation) -- (fitness2);
\draw [arrow] (fitness2) -- (nextgen);
\draw [arrow] (nextgen) -- (decision);
\draw [arrow] (decision) -- ++(0, 2) node[midway, text=black, draw=none, fill=none]{否} -| (selection);
\draw [arrow] (decision) -- node[draw=none, fill=none, anchor=south] {是} (output);

\end{tikzpicture}
\caption{遺傳演算法流程圖}
\label{fig:GA_flowchart}
\end{figure}
% --------------------------
\end{enumerate}
\subsection{選擇方式}
\subsubsection{Roulette Wheel Selection}
個體被選中的概率與其適應度成正比,因此適應度越高的個體被選中的可能性越大。
其中f(i)是第i個個體的適應度,N是種群大小。加總起來得到適應度總和S。
\begin{equation} \label{eq:RouletteWheelSelection}
    S = \sum_{i=1}^{N}f(i)
\end{equation}
每個個體被選重的機率為
\begin{equation}
    P(i) = \frac{f(i)}{S}
\end{equation}
生成隨機數$r \in [0,1)$, 根據累積分佈選擇對應的個體。重複直到選擇出所需數量的個體。
\subsubsection{Tournament Selection}
該方法不直接依賴適應度比例,而是通過隨機競爭進行選擇。以下是選擇步驟:
\begin{enumerate}
    \item \textbf{隨機選取子集:}從種群中隨機選擇 kk 個個體(通常稱為子集或比賽參賽者), kk 通常為 2 或 3。
    \item \textbf{比較適應度:}在子集中選擇適應度最高的個體作為比賽的「贏家」。
    \item \textbf{重複選擇:}重複上述過程,直到選擇出所需數量的個體。
\end{enumerate}

\subsection{交配方式}
\subsubsection{Single-point Crossover}
選擇在父代基因串中的一個位置作為交叉點,並交換兩個父代基因串在該位置的後部分來生成兩個新的子代。
\begin{enumerate}
    \item \textbf{選擇交叉點:}在父代基因串中隨機選擇一個交叉點(假設是第i位)。
    \item \textbf{交換基因:}將父代1的基因從交叉點 i 之後的部分與父代2的基因從交叉點 i 之後的部分互換。
    \item \textbf{生成子代:}由此生成兩個子代個體,每個子代基因的前半部分來自一個父代,後半部分來自另一個父代。
\end{enumerate}
\subsubsection{Multi-point Crossover}
多點交叉是單點交叉的擴展,它在基因串中選擇多個交叉點,並在這些點之間交換基因段。這樣做的目的是進一步增加基因組合的多樣性。
\begin{enumerate}
    \item \textbf{選擇多個交叉點:}在基因串中隨機選擇 kk 個交叉點(通常為 2 或 3 個)。
    \item \textbf{交換基因段:}將每個父代的基因段在交叉點之間進行交換。交換的區段不再是單一位置之後的基因,而是跨越多個區域。
    \item \textbf{生成子代:}根據選定的交叉點及交換規則生成兩個子代。
\end{enumerate}



\subsection{GA程式範例}
以下程式碼使用Genetic Algorithm來尋找一個與目標字符串"HELLO"相符的字串。
\begin{minted}[linenos, breaklines]{python}
import random

# 目標字符串
TARGET = "HELLO"
POPULATION_SIZE = 100
MUTATION_RATE = 0.01
GENERATIONS = 1000

# 隨機生成一個字符串
def random_string(length):
    return ''.join(random.choice('ABCDEFGHIJKLMNOPQRSTUVWXYZ ') for _ in range(length))

# 計算適應度:目標字符串與當前字符串相符的字符數
def fitness(individual):
    return sum(1 for a, b in zip(individual, TARGET) if a == b)

# 選擇:輪盤選擇法
def select(population):
    total_fitness = sum(fitness(ind) for ind in population)
    pick = random.uniform(0, total_fitness)
    current = 0
    for individual in population:
        current += fitness(individual)
        if current > pick:
            return individual

# 交配:單點
def crossover(parent1, parent2):
    point = random.randint(1, len(parent1) - 1)
    child = parent1[:point] + parent2[point:]
    return child

# 突變:以一定機率隨機改變基因
def mutate(individual):
    return ''.join(
        char if random.random() > MUTATION_RATE else random.choice('ABCDEFGHIJKLMNOPQRSTUVWXYZ ')
        for char in individual
    )

def genetic_algorithm():
    # 初始化種群
    population = [random_string(len(TARGET)) for _ in range(POPULATION_SIZE)]
    for generation in range(GENERATIONS):
        # 計算當前最優解
        best_individual = max(population, key=fitness)
        if fitness(best_individual) == len(TARGET):
            print(f"Generation {generation}: Found solution -> {best_individual}")
            return
        print(f"Generation {generation}: Best -> {best_individual}, Fitness -> {fitness(best_individual)}")
        
        # 新一代種群
        new_population = []
        for _ in range(POPULATION_SIZE):
            parent1 = select(population)
            parent2 = select(population)
            child = crossover(parent1, parent2)
            child = mutate(child)
            new_population.append(child)
        population = new_population
    print("No solution found.")

# 執行基因演算法
genetic_algorithm()
\end{minted}

\section{粒子群優化演算法}
粒子群優化演算法(Particle Swarm Optimization, PSO)是一種基於群體智慧的隨機優化方法,模仿了鳥群或魚群在尋找食物時的群體行為。通過模擬個體(粒子)在搜索空間中的運動,來尋找問題的最優解。PSO 是一種啟發式全局優化算法,它並不需要了解問題的具體數學模型,而是通過群體中粒子的協同合作來達到全局最優解。
\subsection{基本原理}
在PSO中,每個粒子具有以下屬性:
\begin{itemize}
    \item 位置(Position):表示粒子在解空間中的位置,對應一個解。
    \item 速度(Velocity):表示粒子在解空間中的移動速度。
    \item 個體最優解(Personal Best, $p_{best}$​):粒子在搜索過程中找到的最好的解。
    \item 全局最優解(Global Best, $g_{best}$):整個粒子群中所有粒子找到的最好的解。
\end{itemize}

\subsubsection{向量更新方程式}
速度向量更新方程式如下,其中$c_1$, $c_2$為學習因子,$r_1$, $r_2$為隨機數,$w$為慣性權重。
\begin{equation} \label{eq:PSO_velocity}
    v_{i}^{(t+1)} = w \cdot v_{i}^{(t)} + c_{1}r_{1}(p_{best}, i - x_{i}^{(t)}) + c_{2}r_{2}(g_{best} - x_{i}^{(t)})
\end{equation}
粒子位置的更新方程式如下:
\begin{equation} \label{eq:PSO_position}
    x_{i}^{(t+1)} = x_{i}^{(t)} + v_{i}^{(t+1)}
\end{equation}

\subsubsection{收斂條件}
\begin{enumerate}
    \item 最大迭代次數:當算法達到預設的最大迭代次數時,停止算法。這是最常見的收斂條件。
    \item 適應度值變化極小:如果在多次迭代中,最佳粒子的適應度值變化非常小,說明已經接近最優解,可以停止算法。
    \item 群體適應度收斂:當粒子群的適應度值接近穩定,且無法再找到更好的解時,視為收斂。
\end{enumerate}


\subsection{PSO步驟}
\begin{enumerate}
    \item \textbf{初始化:}初始化粒子的位置和速度,通常隨機分配。每個粒子的個體最優解設為初始位置,並計算其適應度。
    \item \textbf{更新粒子位置和速度:}根據速度更新公式和位置更新公式,更新每個粒子的位置和速度。
    \item \textbf{更新個體最優解:}如果某個粒子的新位置比它的個體最優解更好,則更新它的個體最優解。
    \item \textbf{更新全局最優解:}檢查所有粒子的個體最優解,選擇最好的解作為全局最優解。
    \item \textbf{重複運算:}重複步驟 2 至 4,直到達到終止條件(如最大迭代次數或達到所需的最佳解)。
\end{enumerate}

\subsection{PSO程式範例}
以下使用PSO用於求解一個簡單的最小化函數問題,目標是找到函數$f(x,y) = x^2 + y^2$的最小值。

\begin{minted}[linenos, breaklines]{python}
import random

# 定義目標函數
def objective_function(position):
    x, y = position
    return x**2 + y**2  # 最小化目標函數

# 粒子類別
class Particle:
    def __init__(self, bounds):
        self.position = [random.uniform(bound[0], bound[1]) for bound in bounds]  # 初始化位置
        self.velocity = [random.uniform(-1, 1) for _ in bounds]  # 初始化速度
        self.best_position = self.position[:]  # 個體最佳位置
        self.best_value = objective_function(self.position)  # 個體最佳適應值
        self.value = self.best_value  # 當前適應值

    def update_velocity(self, global_best, w, c1, c2):
        for i in range(len(self.position)):
            r1 = random.random()
            r2 = random.random()
            cognitive = c1 * r1 * (self.best_position[i] - self.position[i])
            social = c2 * r2 * (global_best[i] - self.position[i])
            self.velocity[i] = w * self.velocity[i] + cognitive + social

    def update_position(self, bounds):
        for i in range(len(self.position)):
            self.position[i] += self.velocity[i]
            # 限制位置在邊界內
            if self.position[i] < bounds[i][0]:
                self.position[i] = bounds[i][0]
            if self.position[i] > bounds[i][1]:
                self.position[i] = bounds[i][1]

# PSO 主函數
def particle_swarm_optimization(bounds, num_particles, max_iterations, w, c1, c2):
    # 初始化粒子群
    particles = [Particle(bounds) for _ in range(num_particles)]
    global_best_position = None
    global_best_value = float('inf')

    for iteration in range(max_iterations):
        for particle in particles:
            # 更新粒子適應值
            particle.value = objective_function(particle.position)
            # 更新個體最佳
            if particle.value < particle.best_value:
                particle.best_value = particle.value
                particle.best_position = particle.position[:]
            # 更新全局最佳
            if particle.value < global_best_value:
                global_best_value = particle.value
                global_best_position = particle.position[:]

        # 更新粒子的速度與位置
        for particle in particles:
            particle.update_velocity(global_best_position, w, c1, c2)
            particle.update_position(bounds)

        # 當前代數的結果
        print(f"Iteration {iteration + 1}: Best Value -> {global_best_value}, Best Position -> {global_best_position}")

    return global_best_position, global_best_value

# 參數設置
bounds = [(-10, 10), (-10, 10)]  # x 和 y 的範圍
num_particles = 30  # 粒子數
max_iterations = 100  # 最大迭代次數
w = 0.5  # 慣性因子
c1 = 1.5  
c2 = 1.5  

# 執行PSO
best_position, best_value = particle_swarm_optimization(bounds, num_particles, max_iterations, w, c1, c2)
print(f"Optimal Solution: Position -> {best_position}, Value -> {best_value}")
\end{minted}