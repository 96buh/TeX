% 產生知識圖的命令
% 圖片路徑,圖片名稱,圖片標籤,評分,優點,缺點
% 最後一個參數是 圖片寬度
\newcommand{\knowledgeGraph}[7]{
    \begin{figure}[htbp!]
        \centering
        \fbox{
            \begin{minipage}{0.9\linewidth}
                \centering
                \begin{minipage}{#7 \linewidth} % 固定圖片寬度
                    \includegraphics[width=\linewidth]{#1}
                \end{minipage}
                \caption{#2}
                \label{#3}
                \vspace{5pt}
                \hrule
                \vspace{5pt}
                \textbf{評分(1/10): #4} \\
                \raggedright
                \hangindent=3em
                \textbf{優點:} \\
                    \begin{enumerate}[itemsep=0.01pt]
                      #5
                    \end{enumerate}
                \textbf{缺點:} \\
                    \begin{enumerate}[itemsep=0.01pt]
                      #6
                    \end{enumerate}
                \vspace{5pt}
            \end{minipage}
        }
    \end{figure}
}

% 產生知識圖的命令
% 圖片路徑,圖片名稱,圖片標籤,評分,優點,缺點
% 最後一個參數是 圖片寬度

\newcommand{\textRightGraphLeft}[7]{
    \begin{figure}[htbp!]
        \centering
        \fbox{
            \noindent\begin{minipage}{#7\textwidth}% adapt widths of minipages to your needs
                \includegraphics[width=\linewidth]{#1}
                \caption{#2}
                \label{#3}
            \end{minipage}%
            \hfill%
            \begin{minipage}{0.5\textwidth}\raggedright
                \textbf{評分 (1/10): #4}\\
                \textbf{優點:} #5\\
                \textbf{缺點:} #6\\
            \end{minipage}
        }
    \end{figure}
}
