\chapter{人機協作:使用生成式AI總結和協助撰稿}

\section{生成式AI的虛擬知識}
以下是王道維教授所寫的\href{https://blog.udn.com/dawweiwang/178728006}{文章}內容使用AI總結文本和產生知識圖。
\subsection{前言:問題的層次}
自2022年11月ChatGPT推出以來,高等教育界迅速感受到這一新科技的衝擊。ChatGPT等生成式AI的出現,對傳統教學方法、評量方式和學術標準帶來了挑戰。許多大學紛紛調整課程與教學策略,針對生成式AI的影響進行討論,並試圖制定對策。
2023年GPT-4的推出,進一步加強了這類技術的能力。GPT-4通過多種專業考試並能夠同時處理圖像與文字,使得生成式AI的應用範圍更廣。在此背景下,教育工作者面臨的不僅是技術變革,還有如何重新思考「知識」的定義及學習目標的挑戰。生成式AI的普及將不可避免地影響未來的人才培育與產業結構,尤其是在第四次工業革命的背景下。
\subsection{從教育觀點來區分生成式AI的三大類型功能}
生成式AI的功能繁多,但可以從教育的角度將其分為三大類型,依據使用者的需求來區分功能:\\
\textbf{正確可靠}:此類功能要求AI提供正確且可驗證的知識,這在教育中至關重要。尤其在科學與技術領域,知識的正確性與可靠性是學術核心,因此對這類功能的要求極高。然而,由於AI的生成基於機率模型,有時候會出現錯誤或偏見,這正是教育工作者對生成式AI感到憂慮的主要原因。\\
\textbf{創意豐富}:生成式AI具備創意生成能力,能夠為使用者提供不同的視角與靈感。在藝術、文學、設計等領域,這類功能可以激發學生的創意,幫助他們在學習過程中探索新的思路。但需要注意的是,這類功能的使用也可能帶來抄襲與著作權問題。\\
\textbf{方便實用}:生成式AI的實用性體現在其能夠快速處理日常任務,減少重複性勞動,從而讓學生將精力集中於更具挑戰性的學習內容。這對於高等教育來說是一項有助於提升學習效率的工具。\\
\newpage
在實際應用中,不同學科對這三類功能的需求不盡相同。理工科系可能更多依賴「正確可靠」的功能,而藝術與設計相關科系則可能更注重「創意豐富」的功能。教育者需要根據學科特點和學習目標,靈活應用生成式AI的不同功能。

\subsection{虛擬知識的生成與定義}
在西方哲學傳統中,知識須具備一定的條件才能被認定為可靠的。根據知識論的基本定義,知識應該是與現實相符、可驗證的,並且具備邏輯一致性。生成式AI在模擬人類知識的過程中,往往是通過數據驅動模型來生成答案,但這些答案未必都能滿足嚴格的知識條件。因此,對生成式AI產出的內容應抱有一定的謹慎態度,尤其是在高等教育中。
\par 知識可以分為顯性知識和隱性知識。顯性知識是可以明確表達和傳授的,而隱性知識則是難以明確傳達的,例如個人的經驗和直覺。生成式AI主要依賴於數據驅動的顯性知識,並且很難處理隱性知識。這意味著,在一些依賴於經驗或感知的學習過程中,AI的幫助是有限的。
虛擬知識的定義
隨著生成式AI的普及,出現了一種介於「知識」與「意見」之間的新型態資訊,筆者稱之為「虛擬知識」。虛擬知識是通過AI模型生成的,不一定完全符合傳統意義上的知識條件,但對於學習者來說,卻可能比真實知識更容易獲取並應用。這種虛擬知識的擴展改變了我們對學習與教育的看法。
\subsection{虛擬知識的特色與影響}
介於「意見」與「知識」之間
虛擬知識通常是生成式AI根據大數據推測得出的結論,並非基於真實世界的經驗驗證。因此,它與傳統知識的最大不同在於其可靠性和可驗證性較低,更多地類似於人們的意見。
更加個人化與實用化的知識
生成式AI能根據使用者的需求快速生成個性化內容,這使得知識的傳遞更加即時且實用,能夠滿足學習者在不同情境下的需求。然而,這也引發了對知識本質的質疑:當知識變得如此輕鬆便捷,學習的深度和內化過程是否會被忽略?
研究者、教育者與學習者的相對地位扁平化
生成式AI的應用模糊了傳統知識傳遞中的上下級關係。學生可以通過AI獲取與教師相當甚至更廣泛的知識,這在一定程度上扁平化了教育中的知識結構。
加速跨領域的知識整合
生成式AI能夠從不同學科中快速檢索相關信息並進行整合,這對於跨領域研究和創新具有極大的潛力。教育者應該鼓勵學生使用AI工具來拓展學科間的連結,促進跨領域的知識交流與融合。
\subsection{具體建議:加速AI素養培育的基礎建設}
在地知識與研究資料的公共化與結構化
應加強各地知識資源的結構化,並推動研究資料的公開化。這有助於提高生成式AI在本地化應用中的可靠性,減少對國外數據的依賴。
系統性開發大型語言模型相關的應用
教育系統應積極開發與生成式AI相關的應用,針對不同學科需求定制AI輔助工具,讓AI在教育中發揮更大的作用。
推動以個人志趣為中心的「外掛式學習」
生成式AI的個性化特性可被運用於推動「外掛式學習」,即學習者根據個人興趣和需求,利用AI工具擴展學習的深度與廣度。這種學習方式將有助於增強學生的自主學習能力。
\subsection{總結}
生成式AI的出現為高等教育帶來了機遇與挑戰。從教育者的角度來看,應該根據功能需求將生成式AI分為「正確可靠」、「創意豐富」與「方便實用」三類,以便在不同學科中靈活應用。此外,我們必須重新思考「知識」的定義,並適應由虛擬知識帶來的教育模式變革。最終,透過加強AI素養的基礎建設,推動更多元化的學習模式,我們可以更好地迎接生成式AI時代的到來。

\subsection{清華大學生成式AI文章知識圖}
\knowledgeGraph{images/w5/GENAI_article_taide.png}{清華大學生成式AI文章-TAIDE}{fig:GAI-article-Taide}{8}{\item 重要知識點都有列出來
\item 連接線條清晰,易於追踪關係
}{\item 某些概念群組之間的關聯性不太明確(有一個"沒有"結點)

}{0.7}

\knowledgeGraph{images/w5/GENAI_article_zephyr.png}{清華大學生成式AI文章-Zephyr}{fig:GAI-article-Zephyr}{7}{\item 使用不同顏色區分主題,視覺清晰
}{\item 相較其他兩張圖,包含的概念數量較少
\item 某些分支發展不夠充分
}{0.4}


\knowledgeGraph{images/w5/GENAI_article_llama3.png}{清華大學生成式AI文章-Llama3}{fig:GAI-article-Llama3}{8.5}{\item 放射狀佈局清晰,層次分明
\item 概念分類合理,邏輯性強
}{\item "產生性AI"和"產生性人工智能"應該要是同一個節點
}{0.4}



\section{AI人機協作撰寫新聞稿}
\subsection{文本1: 直接生成}
使用以下文本內容直接產生新聞稿。
\begin{quote}
星期一(10/7)香港大學李志遠教授(IEEE R10 EAC主席)來訪\\
行程規劃如下:\\
1. 12:30到台南火車站(後站)\\
2. 12:50到永福國小停車,然後度小月用餐\\
3. 13:30到永福國小校長室旁邊會議室和學生及校長老師討論\\
4. 14:00前往仁德國小\\
5. 14:30到15:10仁德國小AI台語課程\\
6. 15:10到16:10仁德國小校長室討論\\
7. 16:50回到榮譽校區\\
8. 17:00到17:30參訪OASE實驗室,並且展示機器人聊天\\
9. 18:00杜卡勒用餐\\
10. 19:00前往台南火車站或台南高鐵站
\end{quote}

\textbf{生成結果:} \vspace{0.5em} \par
\fbox{
\begin{minipage}{0.9\linewidth}
香港大學李志遠教授訪問活動報導\\
香港大學李志遠教授,作為國際電機電子工程師學會亞太地區教育委員會主席,近日前來台南進行為期一天的交流訪問,旨在促進教育和技術的合作,特別是在人工智慧和語言教育領域的發展。\\

此次行程從中午開始,李教授首先抵達台南火車站,隨後前往永福國小與當地學生進行互動。在這裡,他與校長及教師展開了深入的討論,分享了他對於AI和台語課程的見解,並強調科技在教育中的應用潛力。\\

李教授接著參加了仁德國小的AI台語課程,這一創新課程旨在提升學生的語言能力,並且引導學生運用科技工具來學習台語,增強語言學習的趣味性和互動性。他對課程內容和學生的參與度表示讚賞,認為這樣的實踐有助於培養學生的語言能力及科技素養。\\

隨後,李教授前往OASE實驗室,觀摩了機器人聊天的展示。實驗室的工作人員展示了最新的機器人技術,李教授與學生們進行了互動,並鼓勵他們探索科技的創新應用。這一活動引起了學生們的濃厚興趣,許多學生踴躍提問,現場氣氛熱烈。\\

李教授的來訪不僅加深了台灣與香港在教育領域的交流,也為未來的合作奠定了基礎。他的觀點和經驗將激發更多關於教育與科技融合的思考與討論,期待未來能有更多類似的合作機會。
\end{minipage}
}


\subsection{文本2: 人機協作} \label{ai-newspaper}
根據人類寫的新聞稿初稿(參考附錄\ref{appendix:HKUni})使用生成式AI進行標題縮短、內容精簡成三段以內。

\textbf{生成結果(AI產生):} \par
\vspace{0.5em}
\fbox{
\begin{minipage}{0.9\linewidth}
\centering
\Large \textbf{香港大學訪問國立臺南大學 推動AI與臺語教育合作} \\ \vspace{1em}

\raggedleft
\scriptsize 
國立臺南大學新聞稿 \\
聯絡人:知識應用暨網路服務研究中心 \\
李健興主任 \\ 
聯絡電話:06-2602573 \\ 
2024.10.07 \\ \vspace{1em}
\raggedright
\normalsize


\hspace{2em}近期,香港大學的李志遠教授,現任國際電機電子工程師協會亞太地區教育委員會主席(IEEE R10 EAC 主席)訪問台南,參觀國立臺南大學、仁德國小與永福國小,推動 AI 技術與臺語教學結合的交流活動。李教授重點探討了臺英語人機共學模式的發展潛力,強調該模式為學生創造了更加個性化和高效的學習環境,並幫助教師減輕工作負擔。\\

\hspace{2em}在永福國小,李教授與校長、教師及學生深入討論,了解 AI 如何提升學習效果。隨後,他參觀了仁德國小的 AI 臺語課程,展示了 AI 與語言學習結合的潛力。李教授期望這一模式能夠推廣至亞洲其他國家,為學生提供先進的 AI 教育體驗,並探討未來合作的可能性。

\hspace{2em}此次訪問不僅促進了台灣與香港的學術交流,也為 AI 在語言教育中的創新應用奠定了基礎。李教授展望 AI 技術在亞洲教育領域的廣泛應用,期待台灣的經驗能夠推動其他國家採用這一創新模式,為亞洲的教育發展注入新動力。
\end{minipage}
}

\subsection{新聞稿知識圖}

\knowledgeGraph{images/w5/HKUniHnM_taide.png}{人機協助新聞稿-TAIDE}{fig:HnM-article-taide}{4}{\item 有把香港大學和臺南大學連接}{\item 不論是中翻英或是英翻中的效果都很差 \item 出現不相關的國家、人}{0.1}