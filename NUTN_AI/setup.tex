\usepackage{graphicx}
\usepackage[UTF8,heading = true]{ctex}
\usepackage{xeCJK}
\usepackage{shorttoc}
\usepackage[hidelinks, linkcolor=black, colorlinks=true, urlcolor=blue, citecolor=black]{hyperref}
\usepackage{fancyhdr}
\usepackage{geometry}
\usepackage{atbegshi}
\usepackage{enumitem}
\usepackage{caption}
\usepackage{amsmath}
\usepackage{tabularray}
\usepackage{array}
\usepackage{fontspec}
% 插入多頁PDF
\usepackage{pdfpages}
% Wrapping text around figures
\usepackage{wrapfig}
% 解除浮動
\usepackage{placeins} 
\usepackage{appendix}
% 讓參考文獻出現在目錄
\usepackage[nottoc,numbib]{tocbibind}
% syllabus
\usepackage{termcal}
% code block
\usepackage{minted}
\setminted{
    fontfamily=Noto Sans CJK TC,
    style=autumn      % theme
}
% 畫圖
\usepackage{tikz}
\usetikzlibrary{shapes.geometric, arrows}


\makeatletter
\AtBeginDocument{\let\book@l@chapter\l@chapter}
\newcommand{\demotechaptersintoc}{%
  \addtocontents{toc}{\let\protect\l@chapter\protect\l@section}%
}
\newcommand{\promotechaptersintoc}{%
  \addtocontents{toc}{\let\protect\l@chapter\protect\book@l@chapter}%
}
% 把目錄的Appendices改成中文
\renewcommand\appendixtocname{附錄}

\setcounter{secnumdepth}{3}
\setcounter{tocdepth}{3} 

\geometry{a4paper, margin=2cm}

\setCJKmainfont[AutoFakeBold=3]{TW-Kai}
% \setCJKmainfont{Noto Sans CJK TC}
\setmainfont{Times New Roman}

\XeTeXlinebreaklocale "zh"             
\XeTeXlinebreakskip = 0pt plus 1pt
\ctexset{
    contentsname = 目錄,
    listfigurename = 圖目錄,
    listtablename = 表目錄,
    figurename = 圖,
    tablename = 表,
    appendixname = ,
}
\renewcommand{\figureautorefname}{圖}
\renewcommand{\appendixautorefname}{附錄}

% 將長表格跨頁時的文字改成中文
\DefTblrTemplate{contfoot-text}{chs}{接下頁\dots}
\SetTblrTemplate{contfoot-text}{chs}
\DefTblrTemplate{conthead-text}{chs}{(續表)}
\SetTblrTemplate{conthead-text}{chs}

% 設定header, footer
\pagestyle{fancy}
\fancyhead{}
\renewcommand{\headrulewidth}{0pt}
\fancyhead[R]{\footnotesize \parbox[t]{\textwidth}{
\raggedleft 人工智慧作業v1.9 \\ Date: 2024/12/02}}

\fancyfoot[C]{\thepage}

\fancypagestyle{plain}{
% 重新定義 plain 頁面樣式,在chapter開始也會顯示
  \fancyhf{} 
  \fancyhead[R]{\footnotesize \parbox[t]{\textwidth}{
  \raggedleft 人工智慧作業v1.9 \\ Date: 2024/12/02}} 
  
  \fancyfoot[C]{\thepage}
}

