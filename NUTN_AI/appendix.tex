

\chapter{版本異動紀錄}


\begin{longtblr}[
    caption = {版本異動紀錄},
    label = {table:version},
]{
    colspec = {|X[c,3]|X[c,3]|X[c,4]|},
    rowspec = {|Q[m]|Q[m]|Q[m]|},
    hlines,vlines,
    rowhead = 1,
    width = 0.8\linewidth
}
\textbf{版本} & \textbf{日期} & \textbf{修改內容} \\ % 標題不需使用 minipage

1.0 & 2024.10.01 & \begin{minipage}[c]{0.3\textwidth}
                        \vspace{10pt}
                        \centering % 內容置中
                        \raggedright % 靠左對齊
                        1. 新增封面 \\ 
                        2. 新增目錄 \\ 
                        \hangindent=1.2em % 設定每行縮排
                        \hangafter=1 % 從第二行開始縮排
                        3. 新增前四章節的內容(第二到四周上課內容)
                        \vspace{10pt}
                    \end{minipage} \\
1.1 & 2024.10.10 & \begin{minipage}[c]{0.3\textwidth}
                        \vspace{10pt}
                        \centering % 內容置中
                        \raggedright % 靠左對齊
                        1. 清華大學生成式AI文章 \\ 
                        2. 香港大學教授新聞稿 \\ 
                        \vspace{10pt}
                    \end{minipage} \\
1.2 & 2024.10.16 & \begin{minipage}[c]{0.3\textwidth}
                        \vspace{10pt}
                        \centering % 內容置中
                        \raggedright % 靠左對齊
                        1. AI新聞稿評分 \\ 
                        2. ELMO, BERT, GPT  \\ 
                        \vspace{10pt}
                    \end{minipage} \\
1.3 & 2024.10.23 & \begin{minipage}[c]{0.3\textwidth}
                        \vspace{10pt}
                        \centering % 內容置中
                        \raggedright % 靠左對齊
                        1. 今周刊臺南大學AI機器人報導知識圖 \\ 
                        2. 新增QCI學習工具內容  \\ 
                        \vspace{10pt}
                    \end{minipage} \\
1.4 & 2024.11.03 & \begin{minipage}[c]{0.3\textwidth}
                        \vspace{10pt}
                        \centering % 內容置中
                        \raggedright % 靠左對齊
                        1. 台英慧虛擬機器人(第八周上課內容) \\ 
                        \vspace{10pt}
                    \end{minipage} \\
1.5 & 2024.11.10 & \begin{minipage}[c]{0.3\textwidth}
                        \vspace{10pt}
                        \centering % 內容置中
                        \raggedright % 靠左對齊
                        1. 虛擬AI機器人不同參數的比較 \\ 
                        2. 新增致謝 \\
                        \vspace{10pt}
                    \end{minipage} \\
1.6 & 2024.11.17 & \begin{minipage}[c]{0.3\textwidth}
                        \vspace{10pt}
                        \centering % 內容置中
                        \raggedright % 靠左對齊
                        1. 新增陽明交通大學演講文章\\ 
                        \vspace{10pt}
                    \end{minipage} \\
1.7 & 2024.11.20 & \begin{minipage}[c]{0.3\textwidth}
                        \vspace{10pt}
                        \centering % 內容置中
                        \raggedright % 靠左對齊
                        1. 新增課程學習重點\\ 
                        2. 新增AI競賽學習內容\\
                        3. 修改期末報告內容\\
                        \vspace{10pt}
                    \end{minipage} \\
\end{longtblr}





\chapter{香港大學李志遠教授參訪(新聞稿初稿)} \label{appendix:HKUni}
\begin{figure}[htbp!]
    \centering
    \includegraphics[scale=0.62]{images/w5/香港大學李志遠教授參訪(新聞稿初稿)-10072024-1.pdf}
\end{figure}


\chapter{南大報導新聞稿} \label{appedix:NUTNnews1}
\begin{quote}
國立臺南大學李健興教授團隊在國科會產學合作計畫的支持下,與國家高速網路與計算中心、真平企業及女媧創造機器人共同推動「TAIDE臺英語共學家庭先導計畫」。TAIDE是臺灣首個本土開發的可信任AI對話引擎,用於推動雙語學習。南大團隊透過該計畫邀請國際電機電子工程師學會亞太地區教育委員會(IEEE R10 EAC)主席李志遠博士蒞臨臺南大學,並走訪臺南市仁德國小及永福國小,深入瞭解生成式AI技術與TAIDE在臺英語教學上的結合與應用潛力。計畫已於10月7日在臺南啟動,並規劃於10月24日推動至全臺多個縣市,探索AI臺英語共學家庭的全新教育模式。

活動首站來到永福國小進行參訪,並與校長楊淑晏及參與計畫的學生,一同討論生成式AI技術如何與傳統臺語教材結合,優化本土語言學習的應用場景,進而提升學生學習成效。隨後前往仁德國小的AI臺語課程,展示該課程如何透過AI技術結合臺語教學,讓學生在與AI互動的過程中學習臺語,不僅增進學習樂趣,亦展現出AI人機共學在語言教育中的應用潛力。

南大李健興教授指出,TAIDE臺英語人機共學模式能為學生提供個人化且高效的語言學習環境,透過AI技術的即時反饋,減輕教師負擔的同時,也幫助學生快速掌握臺語及英語學習內容。此外,參訪團隊與仁德國小校長李培瑜及教務主任劉時斌,就如何將本土語言教育從校園延伸至家庭進行深入討論,並探討未來合作方向,期望這一模式能推廣至亞洲其他國家,讓更多學生受益。

在臺南大學的行程中,陳惠萍校長親自接待李志遠博士,參觀OASE實驗室的AI機器人自動聊天系統,展示AI在人機共學模式下的多元應用,並強調這一技術對亞洲教育創新的重要意義。此次活動為AI技術與語言教育的深度融合奠定基礎,也突顯未來AI科技在亞洲其他國家推廣的潛力。南大團隊期盼臺灣的經驗能成為亞洲教育發展的參考,為更多國家引入生成式AI教育的新機會。
\end{quote}



